\chapter{Introduction} \label{chp:introduction}

In this thesis I address the subject of recommender systems for large textual datasets. Specifically, I will be working with the Corpus of LDS General Conference talks (\textit{CLDSGCT}). This dataset grows at least bi-yearly and presently contains over 10k documents. It is frequently accessed because it is available to the public. The LDS church, consisting of 15,634,199 members (see \url{http://www.mormonnewsroom.org/facts-and-statistics}), asks its lay members to use it as a source of personal study as well as a teaching resource.  Therefore it is important that users are able to locate documents that have meaningful and pertinent connections---by topic if possible. The shear size of the corpus makes it difficult to sift through. Computational helps are available and more are possible.

% Despite the large user base, %Ariel idea
Information retrieval within large sets of textual data such as the \textit{CLDSGCT} is a common problem. Two main approaches to this problem are search and recommendation. Search is when the user enters a text based query and the computer retrieves documents based on that query. An example of this is Google search. The weakness with search-based approaches is that it relies on the user to already have an idea of what is within the dataset and determine the best search query to access the desired information using vocabulary specific to the corpus. The goal of a recommender system (RS) is to generate meaningful recommendations for items or products, either based on content similarity or, in the case of user-specific recommendations, based on content as well as user meta-data which includes profiles, history, ratings, and social network. Real-world operational examples of industry strength recommender systems are suggestions for books on Amazon, or movies on Netflix. ``The design of such recommendation engines depends on the domain and the particular characteristics of the data available'' \citep{Melville2010}. Like query systems, recommender systems have limitations and biases based on the specific algorithms chosen, available data, and domain. Currently users who desire to use the documents found in the \textit{CLDSGCT} can use two websites to help them: \url{LDS.org} and \url{scriptures.byu.edu}. While both sites are similar in that they have a query system, they do differ. At \url{LDS.org} users can perform a search or browse by topic (by year). At \url{scriptures.byu.edu} users can search for talks by using the parameters of year, speaker, and user query. In both instances, user may lack vocabulary needed to find desired talks/information or a user may want to include results for a topic that span multiple years. %Neither site offers an expandable recommender system (even when related talks are listed, they are do not number more than approximately 5 per topic).

I believe the user experience would be improved by a recommender system which, given some starting document, can recommend related documents. The purpose of this thesis is to compare two ways of building a recommender system. They are compared using the catalog coverage metric. The better of the two systems, based on best catalog coverage, we refer to as RelRec. By using LDA, \textit{RelRec} automatically identifies latent topics in talks and uses topics to locate similar talks.

When I started this work, my hypothesis was that that an LDA-based recommender system would have better catalog coverage. My intuition was that since LDA can infer meta-data with small dimensionality--it would remain highly descriptive and summarative of each document. While \emph{TF-IDF} makes no effort to differentiate between word senses and homographs while the token-topic assignments assigned by Collapsed Gibbs Sampling can be seen as a form of disambiguation. Therefore the hypothesis that an LDA-based recommender system would outperform \emph{TF-IDF}-based one seemed sound.

In this work, I show the LDA-based model has greater cataglog coverage than an off-the-shelf TF-IDF RS, for 1-100 recommendations. More than 100 recommendations were not evaluated.
