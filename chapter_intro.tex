\chapter{Introduction}

In this thesis I address the subject of recommender systems for large textual datasets. Specifically I will be working with the Corpus of LDS General Conference talks (\textit{CLDSGCT}). This dataset grows at least bi-yearly and presently contains over 10k documents. It is frequently accessed because members of the LDS church are asked to use it as a source of personal study and as they teach in their lay church. Therefore it is important that users are able to locate documents that have meaningful connections.

Information retrieval within large sets of textual data such as the \textit{CLDSGCT} is a common problem. Two main approaches to this problem are search and recommendation. Search is when the user enters a text based query and the computer retrieves documents based on that query. An example of this is google search. The weakness with search-based approaches is that it relies on the user to already have an idea of what is within the dataset (i.e. vocabulary) and determine the best search query to access the desired information. The goal of a recommender system is to generate meaningful recommendations to a collection of users for items or products that might interest them. Suggestions for books on Amazon, or movies on Netflix, are real-world examples of the operation of industry strength recommender systems. `The design of such recommendation engines depends on the domain and the particular characteristics of the data available' (\citep{Melville2010}). These systems have limitations and biases based on the specific recommending algorithms chosen, available data, and domain. Currently users wishing to use the documents found in the \textit{CLDSGCT} can use two websites to help them: \url{LDS.org} and \url{scriptures.byu.edu}. At \url{LDS.org} users can perform a search or browse by topic within a year. At \url{scriptures.byu.edu} users can search for talks by using the parameters of year, speaker, and user query.

I believe the user experience would be improved by a recommender system which, given some starting document, can recommend related documents. The purpose of this thesis is to compare two ways of building a recommender system. They are compared using the catalog coverage metric. Ultimately, the system we select for our use, which has the best catalog coverage, we call RelRec. By using LDA, RelRec automatically identifies latent topics in talks and uses topics to locate similar talks.
