\chapter{Introduction}

In this thesis I address the subject of recommender systems for large textual datasets. Specifically, I will be working with the Corpus of LDS General Conference talks (\textit{CLDSGCT}). This dataset grows at least bi-yearly and presently contains over 10k documents. It is frequently accessed because it is available to the public. The LDS church, consisting of 15,000,000 members, asks its lay members to use it as a source of personal study as well as a teaching resource. Therefore it is important that users are able to locate documents that have meaningful connections.

% Despite the large user base,
Information retrieval within large sets of textual data such as the \textit{CLDSGCT} is a common problem. Two main approaches to this problem are search and recommendation. Search is when the user enters a text based query and the computer retrieves documents based on that query. An example of this is Google search, but the weakness with search-based approaches is that it relies on the user to already have an idea of what is within the dataset and determine the best search query to access the desired information using vocabulary specific to the corpus. The goal of a recommender system is to generate meaningful recommendations for items or products, either based on content similarity or, in the case of user-specific recommendations, based on content as well as user meta-data which includes profiles, history, ratings, and social network. Suggestions for books on Amazon, or movies on Netflix, are real-world examples of the operation of industry strength recommender systems. ``The design of such recommendation engines depends on the domain and the particular characteristics of the data available'' \citep{Melville2010}. Like query systems, recommender systems have limitations and biases based on the specific algorithms chosen, available data, and domain. Currently users who desire to use the documents found in the \textit{CLDSGCT} can use two websites to help them: \url{LDS.org} and \url{scriptures.byu.edu}. While both sites are similar in that they have a query system, at \url{LDS.org} users can perform a search or browse by topic within a year while at \url{scriptures.byu.edu} users can search for talks by using the parameters of year, speaker, and user query. In both instances, user may lack vocabulary needed to find desired talks/information.

I believe the user experience would be improved by a recommender system which, given some starting document, can recommend related documents. The purpose of this thesis is to compare two ways of building a recommender system. They are compared using the catalog coverage metric. The better of the two systems, based on best catalog coverage, we refer to as RelRec. By using LDA, \textit{RelRec} automatically identifies latent topics in talks and uses topics to locate similar talks.

When we started this work, our hypothesis was that that an LDA-based recommender system would have better catalog. My intuition was that since LDA can infer meta-data with small dimensionality, it would remain highly descriptive of summarative of each document. In other words, LDA models, when appropriate topic counts exist, are very ...... In this work, we show that this is the case for 1-100 recommendations.
