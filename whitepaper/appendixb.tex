\appendix{Tools \& Links}
\label{tools+links}

\section{Open-Source Tools: Docker, Mallet, npm}

\subsection{Docker*}
Now that algorithms and metrics have been discussed, we can move on to discussing tools that can be used to create reproducible, robust, systems.

%Core to this project is the use of docker.
Dockers are Linux containers which the docker-engine (i.e. docker daemon) can build and run. They can be thought of as Linux virtual machines in some ways, although they tend to be resource/hardware agnostic since they simply share the kernel, RAM, and other hardware on the host on which they are run. They are considered lightweight since instead of running the full Linux stack for each container, only necessary files are run from within the container. This has made it easy for a new market to emerge where products that are software as a service (SaaS) can be hosted at a lower cost. Container Engine by Google (\url{https://cloud.google.com/container-engine/}) and tutum by Tutum (\url{https://www.tutum.co/}) are examples of this.

Docker has a growing community around it. In fact, Google has created an open-source project called Kubernetes which aims to ``accelerate Dev and simplify Ops'' \citep{kube_website}. According to \citet{google_container_engine}, it is used under the covers for Container Engine, a Google cloud product. Although Linux containers have been around for some time now, the docker community has made it much more mainstream, especially for computing environments where scaling horizontally is important. Thus, companies such as Netflix and EMC find it highly beneficial. In the authors experience, Linux tools with growing communities are well backed, become robust, and tend to be both easy to use as well as empowering.

\subsection{Mallet Toolbox}
Mallet toolbox

\subsection{Node Package Manager (NPM)}
 \url{https://www.npmjs.com/}

% Consider not using comma-delimiting
\begin{center}
	\begin{tabular}[pos]{| l | l | l |}
		\hline
		Tool/Library & Use & Link \\ \hline
		git & code versioning; code sharing & \url{https://git-scm.com/about} \\ \hline
		Latex, Google Docs, Microsoft Office & Composing this document & \url{https://www.latex-project.org/}, \url{https://docs.google.com}, \url{https://products.office.com/en-us/word} \\ \hline
		AntConc & Indexing of files to locate high-frequency words later used as stop words. & \url{http://www.laurenceanthony.net/software/antconc/} \\ \hline
		``docker'' (docker daemon, docker-compose, docker containers) & Containerization and modularization of each module such that when running, code for each module only contains input, the necessary code, output, and operating system tools. & \url{https://www.docker.com/} \\ \hline
		mysql SCI database & Input into first module & \url{link not available} \\ \hline
		mysql server, mysql client & hosting + querying of SCI database & \url{http://dev.mysql.com/} \\ \hline
		GNUMake & orchestration of modules (i.e. the command ‘make thesis’ runs each module that hasn’t been run, making sure to do so in the correct order & \url{https://www.gnu.org/software/make/} \\ \hline
		nodeJS, sh, bash, zsh & orchestration of modules (i.e. the command ‘make thesis’ runs each module that hasn’t been run, making sure to do so in the correct order & \url{https://nodejs.org/en/}, \url{https://www.gnu.org/software/bash/} \\ \hline
		VirtualBox & running docker daemon via boot2docker (Linux VM) & \url{https://www.virtualbox.org/} \\ \hline
		Atom, vi, nano & code editors & \url{https://atom.io/}, \url{http://www.vim.org/}, \url{http://www.nano-editor.org/} \\ \hline
		mallet & indexing of files in preparation for LDA; Gibbs Sampling to infer parameters of LDA topic model & \url{http://mallet.cs.umass.edu/} \\ \hline
		Mac OS-X & Host system running VM, hosting code & \url{http://www.apple.com/osx/} \\ \hline
		Bitbucket & code hosting & \url{https://bitbucket.org/} \\ \hline
		GitHub & code hosting location of open-source tools used in the project including nodeJS and mallet & \url{https://github.com/} \\ \hline
		less, more, cat, wc,  grep, egrep, z, oh-my-zsh & general modification or searching of files in *nix & \url{https://en.wikipedia.org/wiki/Less_(Unix)}, \url{http://linux.die.net/man/1/more}, \url{http://linux.die.net/man/1/grep}, \url{http://linux.die.net/man/1/egrep}, \url{http://linux.die.net/man/1/wc}, \url{https://github.com/rupa/z}, \url{https://github.com/robbyrussell/oh-my-zsh} \\ \hline
		regular expressions & easy removal of boiler plate HTML from documents & \url{https://en.wikipedia.org/wiki/Regular_expression} \\ \hline
		npm & hosting of open-source code used in nodeJS & \url{https://www.npmjs.com/} \\ \hline
		\hline
	\end{tabular}
\end{center}
