%TODO: Remove extra space between "Abstract" and title--official template says not to have spacing there.

Having performed prior research on text mining on religious documents, it was made apparent that recommendation systems in the area are lacking if not non-existent for some datasets. For example, both www.LDS.org and www.scriptures.byu.edu are websites built for religious use\footnote{In their case, the members of the Church of Jesus Christ of Latter-day Saints.}. Although they provide users the ability to \textbf{search} for documents based on keywords, they do not provide the ability to \textbf{discover} documents based on similarity. We believe that these systems would greatly benefit from a recommendation system.

This work provides a recommendation system fit for use in either website since they use the same documents in their corpora. Such a recommendation system would provide users another way to explore and discover documents related to their current interests, given a starting document. We call our system \textit{RelRec}. The recommendation system is intuitive because it only takes into account the topic content of each document by using Gibbs-Sampled-Inferred LDA, k-NN (\textit{k}=5) to locate related documents makes the system intuitive. This means that while other recommendation systems would be difficult to describe, our system can be described simply: Documents are considered relevant if they are among the nearest neighbors, where ``nearest'' means that they share a similar distribution over topics. %``The following 5 documents are recommended because they match share similar topics with this one.''

Furthermore, we compare output of \textit{RelRec} with that of a well-accepted TF-IDF recommendation system. This lends credibility to the results and show that either system would be fit for use in these systems. \textit{RelRec} does/doesn't\footnote{Research has not yet been completed.} outperform the TF-IDF system in terms of coverage and serendipity, making it robust and preferable.
