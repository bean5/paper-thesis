%TODO: Remove extra space between "Abstract" and title--official template says not to have spacing there.

Through prior research on text mining of religious documents, it became apparent that recommendation systems in the area are lacking, if not non-existent, for some datasets. For example, both \url{www.LDS.org} and \url{scriptures.byu.edu} are websites designed for religious use. Although they provide users with the ability to search for documents based on keywords, they do not provide the ability to discover documents based on similarity. Consequently, these systems would greatly benefit from a recommendation system.

This work provides a recommendation system fit for use in either website since both use the same documents in their corpora. Such a recommendation system would provide users another way to explore and discover documents related to their current interests, given a starting document. The recommendation system created for this thesis, \emph{RelRec}, is intuitive because it only takes into account the topic content of each document by using Gibbs-Sampled-Inferred LDA, k-NN (k=5) to locate related documents makes the system intuitive. This means that while other recommendation systems would be difficult to describe, our system can be described simply: Documents are considered relevant if they are among the nearest neighbors, where ``nearest'' means that they share a similar distribution over topics. Furthermore, we compare output of \emph{RelRec} with that of a well-accepted \emph{TF-IDF} recommendation system. This lends credibility to the results and show that either system would be fit for use in these systems.

We show that \emph{RelRec} outperform the \emph{TF-IDF} system in terms of coverage, making it robust and preferable for our problem and data domain. %TODO Mention that it does/doesn't outperform in terms of serendipity.
